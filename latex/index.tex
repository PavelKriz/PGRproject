\hypertarget{index_Zadání}{}\section{Zadání}\label{index_Zadání}
Jde o smyšlený svět s planetou Hawaii. Samotná planeta se skládá z vesmírného ananasu. Kolem planety se otáčí mohutný prstenec pizzového typu. Jelikož je vesmírný ananas mladá a žhavá planeta, tak vyzařuje svoji ananasovou auru ve svém okolí. Také z jeho rozžhaveného jádra vystřelují kousky ananasů. Některé jsou ze samotného středu planety a tak dokonce ještě žhnou. Všechny kousky ananasů nemají však dostatečnou rychlost a tak vždy skončí jen na prstenci planety.

Jako divák této scenérie máte na výběr podle toho, zda jste zastánce ananasu na pizze nebo ne. Jestli ne , tak se můžete snažit ananas z pizzy odstranit klikáním na ananasové kousky. Prozradím však, že je to boj s větrnými mlýny, ananasy jen tak nedojdou. Pokud máte rádi ananas na pizze můžete pozorovat scenérii, jak kolem planety obíhá slunce, nebo padající kousky ananasu. Rychlost generace kousků si může štelovat funkčními klávesami nebo klikáním na ananas generovat kousky. Také si na lepší prozkoumání může zapnout kameru, nebo kliknutím na pizzu tento prstenec zastavovat nebo roztáčet.

Kolem planety se dá libovolně pohybovat, nikdy se však nedostanete daleko do hlubokého vesmíru.\hypertarget{index_Ovládání}{}\section{Ovládání}\label{index_Ovládání}
W / Sipka nahoru -\/ posun diváka kupředu za nosem

S / Sipka dolu -\/ odsun diváka přímo v opačném směru než za nosem

A / Sipka doleva -\/ odsun diváka doleva

D / Sipka doprava -\/ odsun diváka doprava

Q -\/ vzestup diváka vzhůru

E -\/ sestup diváka dolů

1 -\/ volný pohled

2 -\/ 1. statický uzamknutý pohled

3 -\/ 2. statický uzamknutý pohled

F1 až F4 -\/ rychlost generace kousků ananasů -\/ od nejpomalejší F1 k nejrychlejší F4

l -\/ zapnutí baterky(reflektoru)

Klikání a drag probíhá jakýmkoliv ze tří hlavních tlačítek myši.

Drag a pohyb myší -\/ rozhlížení se kolem

Kliknutí myší na Ananas -\/ ananas vygeneruje kousek ananasu

Kliknutí myší na pizzu -\/ zastaví nebo roztočí pizzu

Kliknutí myší na kousek ananasu -\/ odstraní kousek ananasu

E\+SC -\/ ukončení programu\hypertarget{index_Technologie}{}\section{Technologie}\label{index_Technologie}
c++14, glsl v330

pgr -\/ P\+GR library -\/ contains some useful functions -\/ \href{https://cent.felk.cvut.cz/courses/PGR/framework.html}{\texttt{ https\+://cent.\+felk.\+cvut.\+cz/courses/\+P\+G\+R/framework.\+html}}

F\+R\+E\+E\+G\+L\+UT 3.\+0.\+0 -\/ The Open\+GL Utility Toolkit -\/ \href{http://freeglut.sourceforge.net/}{\texttt{ http\+://freeglut.\+sourceforge.\+net/}}

gl\+Load\+Gen -\/ Open\+GL Loader Generator -\/ \href{https://bitbucket.org/alfonse/glloadgen/wiki/Home}{\texttt{ https\+://bitbucket.\+org/alfonse/glloadgen/wiki/\+Home}}

G\+LM 0.\+9.\+8.\+5 -\/ Open\+GL Mathematics Library -\/ \href{http://glm.g-truc.net/}{\texttt{ http\+://glm.\+g-\/truc.\+net/}}

Dev\+IL 1.\+8.\+0 -\/ Image Library -\/ \href{http://openil.sourceforge.net/}{\texttt{ http\+://openil.\+sourceforge.\+net/}}

Připojené knihovny

$<$iostream$>$

$<$pgr.\+h$>$

$<$vector$>$

$<$utility$>$

$<$math.\+h$>$ 